% Options for packages loaded elsewhere
\PassOptionsToPackage{unicode}{hyperref}
\PassOptionsToPackage{hyphens}{url}
%
\documentclass[
]{article}
\usepackage{amsmath,amssymb}
\usepackage{lmodern}
\usepackage{ifxetex,ifluatex}
\ifnum 0\ifxetex 1\fi\ifluatex 1\fi=0 % if pdftex
  \usepackage[T1]{fontenc}
  \usepackage[utf8]{inputenc}
  \usepackage{textcomp} % provide euro and other symbols
\else % if luatex or xetex
  \usepackage{unicode-math}
  \defaultfontfeatures{Scale=MatchLowercase}
  \defaultfontfeatures[\rmfamily]{Ligatures=TeX,Scale=1}
\fi
% Use upquote if available, for straight quotes in verbatim environments
\IfFileExists{upquote.sty}{\usepackage{upquote}}{}
\IfFileExists{microtype.sty}{% use microtype if available
  \usepackage[]{microtype}
  \UseMicrotypeSet[protrusion]{basicmath} % disable protrusion for tt fonts
}{}
\makeatletter
\@ifundefined{KOMAClassName}{% if non-KOMA class
  \IfFileExists{parskip.sty}{%
    \usepackage{parskip}
  }{% else
    \setlength{\parindent}{0pt}
    \setlength{\parskip}{6pt plus 2pt minus 1pt}}
}{% if KOMA class
  \KOMAoptions{parskip=half}}
\makeatother
\usepackage{xcolor}
\IfFileExists{xurl.sty}{\usepackage{xurl}}{} % add URL line breaks if available
\IfFileExists{bookmark.sty}{\usepackage{bookmark}}{\usepackage{hyperref}}
\hypersetup{
  hidelinks,
  pdfcreator={LaTeX via pandoc}}
\urlstyle{same} % disable monospaced font for URLs
\usepackage[margin=1.0in]{geometry}
\usepackage{graphicx}
\makeatletter
\def\maxwidth{\ifdim\Gin@nat@width>\linewidth\linewidth\else\Gin@nat@width\fi}
\def\maxheight{\ifdim\Gin@nat@height>\textheight\textheight\else\Gin@nat@height\fi}
\makeatother
% Scale images if necessary, so that they will not overflow the page
% margins by default, and it is still possible to overwrite the defaults
% using explicit options in \includegraphics[width, height, ...]{}
\setkeys{Gin}{width=\maxwidth,height=\maxheight,keepaspectratio}
% Set default figure placement to htbp
\makeatletter
\def\fps@figure{htbp}
\makeatother
\setlength{\emergencystretch}{3em} % prevent overfull lines
\providecommand{\tightlist}{%
  \setlength{\itemsep}{0pt}\setlength{\parskip}{0pt}}
\setcounter{secnumdepth}{-\maxdimen} % remove section numbering
\usepackage{palatino}
\usepackage{setspace}
\doublespacing
\usepackage[left]{lineno}
\linenumbers
\usepackage{indentfirst}
\setlength{\parindent}{20pt}
\ifluatex
  \usepackage{selnolig}  % disable illegal ligatures
\fi
\newlength{\cslhangindent}
\setlength{\cslhangindent}{1.5em}
\newlength{\csllabelwidth}
\setlength{\csllabelwidth}{3em}
\newenvironment{CSLReferences}[2] % #1 hanging-ident, #2 entry spacing
 {% don't indent paragraphs
  \setlength{\parindent}{0pt}
  % turn on hanging indent if param 1 is 1
  \ifodd #1 \everypar{\setlength{\hangindent}{\cslhangindent}}\ignorespaces\fi
  % set entry spacing
  \ifnum #2 > 0
  \setlength{\parskip}{#2\baselineskip}
  \fi
 }%
 {}
\usepackage{calc}
\newcommand{\CSLBlock}[1]{#1\hfill\break}
\newcommand{\CSLLeftMargin}[1]{\parbox[t]{\csllabelwidth}{#1}}
\newcommand{\CSLRightInline}[1]{\parbox[t]{\linewidth - \csllabelwidth}{#1}\break}
\newcommand{\CSLIndent}[1]{\hspace{\cslhangindent}#1}

\title{\LARGE

The contribution of genetic and environmental effects to Bergmann's rule
and Allen's rule in house mice}
\usepackage{etoolbox}
\makeatletter
\providecommand{\subtitle}[1]{% add subtitle to \maketitle
  \apptocmd{\@title}{\par {\large #1 \par}}{}{}
}
\makeatother
\subtitle{\vspace{10mm}\large

Mallory A. Ballinger and Michael W. Nachman}
\author{}
\date{\vspace{-2.5em}}

\begin{document}
\maketitle

\vspace{-10mm}

\noindent Department of Integrative Biology

\noindent Museum of Vertebrate Zoology

\noindent University of California, Berkeley

\noindent Berkeley, CA 94702-3160

\vspace{5mm}

\noindent To whom correspondence should be addressed:

\noindent\href{mailto:mallory.ballinger@berkeley.edu}{mallory.ballinger@berkeley.edu},
\noindent\href{mailto:mnachman@berkeley.edu}{mnachman@berkeley.edu}

\vspace{10mm}

\noindent\textbf{Running title:} Ecogeographic rules in house mice

\noindent\textbf{Keywords:} body size, extremity length, adaptive
plasticity, heritability, \emph{Mus}

\noindent\textbf{Article type:} Major article

\newpage

\hypertarget{abstract}{%
\subsection{Abstract}\label{abstract}}

\noindent Distinguishing between genetic, environmental, and
genotype-by-environment effects is central to understanding geographic
variation in phenotypic clines. Two of the best-documented phenotypic
clines are Bergmann's rule and Allen's rule, which describe larger body
sizes and shortened extremities in colder climates, respectively.
Although numerous studies have found inter- and intraspecific evidence
for both ecogeographic patterns, we still have a poor understanding of
the extent to which these patterns are driven by genetics, environment,
or both. Here, we measured the genetic and environmental contributions
to Bergmann's rule and Allen's rule across introduced populations of
house mice (\emph{Mus musculus domesticus}) in the Americas. First, we
documented clines for body mass, tail length, and ear length in natural
populations, and found that these conform to both Bergmann's rule and
Allen's rule. We then raised descendants of wild-caught mice in the lab
and showed that these differences persisted in a common environment and
are heritable, indicating that they have a genetic basis. Finally, using
a full-sib design, we reared mice under warm and cold conditions. We
found very little plasticity associated with body size, suggesting that
Bergmann's rule has been shaped by strong directional selection in house
mice. However, extremities showed considerable plasticity, as both tails
and ears grew shorter in cold environments. These results indicate that
adaptive phenotypic plasticity as well as genetic changes underlie major
patterns of clinal variation in house mice and likely facilitated their
rapid expansion into new environments across the Americas.

\newpage

\hypertarget{introduction}{%
\subsection{Introduction}\label{introduction}}

Clines in phenotypes have historically been attributed to natural
selection, reflecting adaptation to local environments (Huxley 1939;
Endler 1977). Two of the best described clinal patterns in animals are
Allen's rule and Bergmann's rule. Allen's rule is the observation that
extremities, such as limb length and tail length, are shorter in colder
climates compared to warmer regions, resulting in latitudinal clines
(Allen 1877). Bergmann's rule is the observation that body sizes are
larger in colder climates, resulting in latitudinal clines in body size
(Bergmann 1847). Shortened extremities and larger body sizes minimize
heat loss by reducing surface area to volume ratios and are thus viewed
as thermoregulatory adaptations (Mayr 1956). Numerous studies have
documented Bergmann's rule and Allen's rule within and across species of
birds (Johnston and Selander 1964; James 1970; Laiolo and Rolando 2001;
Romano et al. 2020) and mammals (Brown and Lee 1969; Griffing 1974;
Yom-Tov and Nix 1986; Fooden and Albrecht 1999), including humans (Ruff
1994; Ruff 2002; Foster and Collard 2013; Betti et al. 2015). Moreover,
various meta-analyses have supported the generality of these rules
(Ashton et al. 2000; Ashton 2002; Freckleton et al. 2003; Meiri and
Dayan 2003; Blackburn and Hawkins 2004; Millien et al. 2006; Nudds and
Oswald 2007; Olson et al. 2009; Symonds and Tattersall 2010; Alhajeri et
al. 2020). On the other hand, several meta-analyses have questioned the
ubiquity of these patterns arguing that statistical support is weak
(Geist 1987; Gohli and Voje 2016; Riemer et al. 2018) or that phenotypic
differences are more likely to be driven by resource abundance or other
factors rather than by considerations of temperature (Scholander 1955;
McNab 1971; Geist 1987; Alhajeri and Steppan 2016; Alroy 2019). The
contradicting results found across the literature are unsurprising given
the variation within and among datasets, such as choice of taxonomic
groups, environmental variables, and inconsistencies in measurements.
Moreover, virtually all studies to date are based on observations of
individuals sampled in natural populations in which factors such as age,
reproductive condition, social status, pathogen and parasite loads, and
overall health are not easily controlled. Thus, we still have very
little understanding of the mechanisms underlying Allen's rule and
Bergmann's rule.

Missing from many of these discussions are careful analyses determining
which traits are genetically encoded, environmentally influenced, or
both. Environmentally influenced traits are phenotypically plastic
traits, and these traits may also harbor genetic variation for
plasticity (i.e., genotype-by-environment interactions)(Des Marais et
al. 2013). Most traits associated with Bergmann's rule and Allen's rule
are complex, meaning they are both polygenic and strongly influenced by
the environment (Falconer and Mackay 1996; Lynch et al. 1998; Yang et
al. 2010; Harpak and Przeworski 2021). In fact, in his original
description of clinal patterns, Allen (1877) emphasized the role of the
environment in directly modulating phenotypes. Disentangling genetic
from non-genetic effects in natural populations is difficult when using
phenotypic data collected from wild-caught animals. Genetic
contributions to trait values may be masked by environmental effects
(plasticity) and genotype-by-environment interactions (Conover and
Schultz 1995; Alho et al. 2011). Phenotypic plasticity may also generate
clinal patterns, giving a false impression of adaptive clines (James
1983). In fact, many temporal changes in body size are driven by the
environment and not genetic adaptation in birds (Teplitsky et al. 2008;
Husby et al. 2011) and mammals (Ozgul et al. 2009, 2010). Furthermore,
we have little understanding of how populations conforming to these
ecogeographic rules vary in the degree and direction of plasticity they
exhibit in response to environmental stimuli. Variation in plasticity
(i.e., genotype-by-environment interactions) may facilitate adaptation
and divergence in polygenic traits (Via and Lande 1985; Gillespie and
Turelli 1989; Gomulkiewicz and Kirkpatrick 1992; West-Eberhard 2003).
However, controlling for environmental effects and measuring the
contributions of phenotypic plasticity is difficult, as transplant
experiments and common garden experiments are infeasible for many taxa.
These limitations have impeded our ability to make substantial progress
on understanding the evolutionary and ecological mechanisms underlying
Bergmann's rule and Allen's rule.

House mice (\emph{Mus musculus domesticus}) provide a tractable system
for disentangling the genetic and environmental contributions to complex
traits. House mice have recently expanded their range from Western
Europe to the Americas, where they can be found from the tip of South
America to Alaska. Across this broad latitudinal range, mice are exposed
to various environmental gradients, including both temperature and
aridity (Phifer-Rixey and Nachman 2015). Despite residing in these novel
environments for only a few hundred generations, there is evidence for
clinal variation across latitudes. Specifically, mice in eastern North
America follow Bergmann's rule (Lynch 1992; Phifer-Rixey et al. 2018),
with larger mice in more northern populations. These body size
differences persist in a common environment over several generations,
indicating that they have a genetic basis (Lynch 1992; Phifer-Rixey et
al. 2018). Selection on house mice over ten generations in the
laboratory recapitulates these clinal patterns: mice bred at lower
temperatures become larger and undergo genetic divergence in body size
(Barnett and Dickson 1984). Furthermore, previous work has revealed an
environmental influence on tail length when exposed to cold
temperatures. Specifically, house mice reared in a cold environment grew
significantly shorter tails than mice reared at warm temperatures,
consistent with Allen's rule (Sumner 1909, 1915; Barnett 1965). However,
these earlier studies investigated only a single population of mice or
used classical inbred laboratory strains of mice, making it difficult to
place the results in an explicit evolutionary framework. We still have
little understanding of the phenotypic variation of house mice across
their entire latitudinal distribution, and even less understanding of
the contributions of genetics and environment to these complex traits.

Here, we use a combination of approaches to tease apart genetics from
plasticity in Bergmann's rule and Allen's rule in house mice from North
and South America. First, we determined if house mice conform to both
Bergmann's rule and Allen's rule across North and South America by
analyzing phenotypic data from wild-caught individuals. Second, we
collected individuals from temperate and tropical populations of house
mice from the ends of their latitudinal distribution, brought them back
to the lab, and established wild-derived colonies. We analyzed
phenotypic differences between populations and across generations in a
common lab environment and identified a genetic basis to both Allen's
rule and Bergmann's rule. Third, to measure the influence of environment
on body size and extremity length, we performed a second common garden
experiment by rearing both lab populations of mice using a full-sib
design in a cold and warm environment and measured the effects on body
size and extremity length. Measuring developmental plasticity within and
between populations allowed us to assess the influence of temperature on
complex traits and to understand the evolutionary mechanisms underlying
these clinal patterns. Specifically, we found that unlike body size,
tail and ear length are highly plastic and this plastic response goes in
the same direction as the evolved response of New York mice,
highlighting an example of adaptive phenotypic plasticity.

\vspace{5mm}

\hypertarget{materials-and-methods}{%
\subsection{Materials and Methods}\label{materials-and-methods}}

\noindent\emph{Phenotypic data from wild-caught mice}

To determine if house mice conform to Allen's rule and Bergmann's rule,
we tested for associations between body mass, tail length, ear length,
and latitude in wild house mice collected across North and South
America. We downloaded specimen data of all house mouse records from
VertNet (Constable et al. 2010) on October 13, 2020, using the search
query: \emph{vntype}:specimen, \emph{genus}:Mus. We obtained 62,139
museum records and retained records that included \emph{Mus musculus}
specimens collected in North or South America (excluding islands). We
omitted individuals explicitly listed as pregnant, juvenile, subadult,
or immature, and included individuals listed as adult, mature, or with
no age class or reproductive condition specified. We also manually coded
females and males as `adult' if they fulfilled any of the following
criteria: females - presence of placental scars, parous, or lactating;
males - presence of seminal vesicles, testes descended (TD), or testes
scrotal (TS). Tail lengths shorter than 20mm and longer than 120mm
(\emph{n} = 8), and ear lengths greater than 30mm (\emph{n} = 1) were
considered extreme outliers (greater than 3.5 standard deviations from
the mean) and were removed from downstream analyses. Sample information
for the final VertNet dataset (\emph{n} = 3,018) is provided in Data S1.

We assessed the overall relationship between body mass, extremity
length, and latitude by fitting linear models in R (v. 4.1.1), including
sex as a covariate (Table S1). We tested for clinal patterns of body
mass and extremity length across latitude in wild-caught house mice
using Spearman correlations for each sex separately. To test if mice
from colder regions conform to Bergmann's rule and Allen's rule, we
extracted mean annual temperature data for all VertNet sampling
locations at 1 km (30-seconds) spatial resolution from WorldClim 2.1
(Fick and Hijmans 2017). We tested for clinal patterns of body mass and
extremity length across mean annual temperature (\textsuperscript{o}C)
using Spearman correlations for each sex separately.

\vspace{3.5mm}

\noindent\emph{Laboratory-reared mice - common garden experiment 1}

To disentangle genetic effects from environmental effects, we collected
live animals from two locations that represent the ends of a latitudinal
transect: Manaus, Amazonas, Brazil (MAN), located near the equator at
3\textsuperscript{o}S latitude, and Saratoga Springs, New York, USA
(SAR), located at 43\textsuperscript{o}N latitude. Details of this
common garden experiment are given in (Phifer-Rixey et al. 2018) and
(Suzuki et al. 2020). Briefly, live mice from both Brazil (\emph{n} =
38) and New York (\emph{n} = 30) were brought back to the lab at the
University of California, Berkeley. Within each population, unrelated
pairs of wild-caught mice were mated to produce first generation (N1)
lab-reared mice. Mice were then paired in sib-sib matings to generate
inbred lines. These inbred lines (New York: \emph{n} = 10 lines; Brazil:
\emph{n} = 12 lines) have been maintained through sib-sib matings for
over 10 generations. Wild-caught mice and their descendants were housed
in a standard laboratory environment at 21\textsuperscript{o}C with a
12-hr dark and 12-hr light cycle. Water and commercial rodent chow
(Teklad Global, 18\% protein, 6\% fat) were provided ad libitium.
Standard museum measurements (total length, tail length, hind foot
length, ear length, and body mass) were taken for all wild-caught and
inbred mice from each population (see Data S2). We removed outliers for
tail length (\textless{} 50mm (\emph{n} = 2)) and ear length
(\textless{} 8mm (\emph{n} = 1)) from downstream analyses.

\vspace{3.5mm}

\noindent\emph{Estimating heritability}

To estimate heritability for body mass and extremity length in house
mice, we performed midparent-offspring regression on N2 (parents) and N3
(offspring) laboratory-born mice. Generations N2 and N3 were chosen to
eliminate any residual environmental and maternal effects that could
influence heritability estimates. Midparent values were calculated as
the mean trait value between mother and father, and heritabilities were
calculated as the regression coefficients (slopes) of osspring values
against midparent values (Lynch et al. 1998). We performed
midparent-offspring regressions on 13 Brazil families (representing 10
different inbred lines) and 13 New York families (representing 10
different inbred lines). We also calculated regression coefficients of
offspring values against maternal and paternal values separately for all
three traits. Heritabilities from these regressions were estimated as
twice the slope of the regression of offspring values against maternal
or paternal values (Lynch et al. 1998). We assessed the significance of
regression coefficients for each heritability estimate using ANOVAs,
implemented in the car package (v. 3.0.11) (Fox and Weisberg 2019).

\vspace{3.5mm}

\noindent\emph{Developmental phenotypic plasticity - common garden
experiment 2}

To determine the influence of phenotypic plasticity on body mass and
extremity length, we performed a second common garden experiment by
rearing laboratory-born mice from both populations in a cold and warm
environment. We used temperature as the environmental variable because
temperature is highly correlated with latitude (Millien et al. 2006) and
phenotypic variation in wild house mice across North and South America
is explained most by temperature-related variables (Suzuki et al. 2020).
Specifically, we used two wild-derived inbred lines each from Brazil
(MANA, MANB) and New York (SARA, SARB). Each line has been inbred for
more than 10 generations, and thus mice within a line harbor reduced
levels of genetic variation. Roughly equal numbers of males and females
were produced for each within-line comparison (New York: \emph{n} = 40;
Brazil: \emph{n} = 40; see Data S3 and S4). These population-specific
sample sizes align with previous experimental studies in house mice
(e.g., (Phifer-Rixey et al. 2018)). Full-sibs were born at room
temperature (21\textsuperscript{o}C) and singly-housed at weaning
(\textasciitilde21 days old). After a brief acclimation period, we
randomly assigned 3.5-week-old mice into size-matched groups based on
sex-specific body mass, and then housed mice at either
5\textsuperscript{o}C or remained at 21\textsuperscript{o}C for the
duration of the experiment (\textasciitilde50 days total). We measured
initial body mass and tail length and recorded subsequent body mass and
tail lengths once a week for each mouse. At the end of the experiment,
we euthanized mice at 75 ± 3 days of age, and recorded final body mass
and tail length, in addition to standard museum measurements. Two final
ear lengths were not included in downstream analyses due to ear damage.
We deposited skulls and skeletons of all mice in the Museum of
Vertebrate Zoology, University of California, Berkeley (catalog numbers
are given in Data S4). All experimental procedures were in accordance
with the UC Berkeley Institutional Animal Care and Use Committee
(AUP-2017-08-10248).

\vspace{3.5mm}

\noindent\emph{Data Analysis}

All data analyses and visualizations were completed in R (v. 4.1.1).
Within R, we used the tidyverse (v. 1.3.1) (Wickham et al. 2019),
performance (v. 0.8.0) (Lüdecke et al. 2021), cowplot (v. 1.1.1), here
(v. 1.0.1), and rmarkdown (v. 2.11) (Allaire et al. 2021) packages,
along with R base library. Relative tail length and relative ear length
were calculated by dividing tail or ear length by body mass for each
individual. We also performed all analyses using tail length residuals
and ear length residuals (by regressing length from body mass across
individuals) and obtained similar results.

For common garden experiments 1 and 2, we fitted linear mixed models in
the lme4 R package (v. 1.1.27.1) (Bates et al. 2015) to determine if
morphology varied between sex, population, generation, or environment.
In each model, we included: 1) the morphological trait as the response
variable; 2) sex, population, generation (experiment 1) or environment
(experiment 2), and their interaction as fixed effects; and 3) inbred
lines as a random effect. The significance of interactions was evaluated
using ANOVA based on type III (partial) sums of squares, implemented in
the car package (v. 3.0.11) (Fox and Weisberg 2019). We performed
\emph{post hoc} comparisons on significant two-way interactions using
Tukey's HSD tests (\emph{P} \textless{} 0.05). Lastly, we calculated the
effect size (\(\omega2\)) of predictors using the effectsize package (v.
0.5)(Ben-Shachar et al. 2020) to evaluate the relative influence of sex,
population, and generation or environment on body mass and extremity
length ((Olejnik and Algina 2003); Table 1; Table S2).

The code to perform analyses for this study is available as a git-based
version control repository on GitHub
(\url{https://github.com/malballinger/Ballinger_allenbergmann_AmNat_2021}).
The analysis can be reproduced using a GNU Make-based workflow with
built-in bash tools (v. 3.81) and R (v. 4.1.1).

\vspace{5mm}

\hypertarget{results}{%
\subsection{Results}\label{results}}

\noindent\emph{Evidence for Bergmann's rule and Allen's rule in wild
house mice}

We assessed the relationship between tail length, ear length, body mass,
and latitude in mice collected across North and South America to
determine if populations of house mice conform to Allen's rule and
Bergmann's rule. Using a large dataset downloaded from VertNet (\emph{n}
= 3,018; Data S1), we found little evidence for Bergmann's rule, as body
mass showed a non-significant, positive correlation with latitude across
both males and females (Figure 1A). In contrast, we found stronger
evidence for Allen's rule in house mice from the Americas, with both
tail length (Figure 1C) and ear length (Figure 1E) showing a
significant, negative correlation with latitude. These patterns of
extremity length largely hold true across both sexes (Figure 1C, 1E). In
each of these comparisons, however, latitude explained less than 1\% of
the phenotypic variation (Table S1). Similar patterns of Bergmann's rule
and Allen's rule were seen using mean annual temperature
(\textsuperscript{o}C) as a predictor (Figure S1). Specifically, body
mass declined and extremity length increased with increasing mean annual
temperature (Figure S1A,C,E).

The lack of evidence for Bergmann's rule in Figure 1A and S1A may be due
to the influence of uncontrolled factors (e.g., age, diet, health),
environmental effects, or phenotypic plasticity. Although we minimized
variation in age and reproductive status by removing explicitly labeled
pregnant females, juveniles, and subadults, we still see large variation
across all three traits, likely due to various factors that were not
recorded. To reduce this variation, we further filtered the VertNet
dataset to only include explicitly labeled adult males (Figure 1B, 1D,
1F; \emph{n} = 445). We focused on males since females show more
variation in body mass than males (Levene's test; \emph{F} = 17.89;
\emph{P} = 0.005), likely due to reproductive condition. In this more
controlled set of adult males, we see strong evidence for both
Bergmann's rule (Figures 1B, S1B) and Allen's rule (Figures 1D, 1F, S1D,
S1F). The comparison between the larger dataset and the more curated
dataset highlights how uncontrolled variation in collated museum
metadata may obscure broad ecogeographic patterns.

\vspace{3.5mm}

\noindent\emph{Differences in body mass and extremity length persist in
a common environment and are heritable}

The phenotypic clines observed across wild house mice could represent
genetic differences, phenotypic plasticity, or both. To disentangle
genetics from plasticity, we collected live mice from near the equator
(Manaus, Amazonas, Brazil) and from 43\textsuperscript{o}N latitude
(Saratoga Springs, New York, USA) and brought them into a common
laboratory environment. Population-specific differences in body mass and
extremity length in wild-caught mice (N0) persisted across generations
of laboratory-reared mice (Figure S2). Specifically, mice from New York
were larger than mice from Brazil (ANOVA, \emph{P} \textless{} 0.001,
\(\omega^2\) = 0.56) (Figure S2A; Table S2). New York mice also had
shorter tails (ANOVA, \emph{P} \textless{} 0.001, \(\omega^2\) = 0.65)
(Figure S2B; Table S1) and shorter ears (ANOVA, \emph{P} \textless{}
0.001, \(\omega^2\) = 0.58) (Figure S2C; Table S2) compared to mice from
Brazil. The maintenance of body mass and extremity length differences in
a common environment and across generations suggests that these traits
are heritable in house mice.

To estimate the heritability (\emph{h\textsuperscript{2}}) of body mass
and extremity length in New York and Brazil house mice, we performed
midparent-offspring regressions on N2 and N3 mice (Figure 2). Both
Brazil and New York mice yielded similar, high heritability values for
body mass and extremity length. Specifically, Brazil mice showed
significant and very similar \emph{h\textsuperscript{2}} for all three
traits (ANOVA, \emph{P} \textless{} 0.05; Figure 2A,C,E). These
heritability estimates are in general agreement with previous estimates
of body mass and tail length in house mice (Rutledge et al. 1973).
Furthermore, using single-parent offspring regressions, both New York
and Brazil mice showed similar heritabilities for all three traits
(Figure S3). Maternal- and paternal-offspring regressions suggest that
body mass and extremity length in New York and Brazil mice are not
determined by inheritance of just one sex, since
\emph{h\textsuperscript{2}} estimates are in rough agreement with
midparent-offspring regression estimates. Overall, these results suggest
that body mass and extremity length are heritable and under strong
genetic control in house mice.

\vspace{3.5mm}

\noindent\emph{Extremity length, but not body size, is greatly
influenced by temperature}

The results presented above identified phenotypic divergence in body
mass and tail and ear length in house mice, with New York mice having
shorter tails and ears and larger body sizes than mice from Brazil,
consistent with Allen's rule and Bergmann's rule, respectively. To
determine the influence of phenotypic plasticity on these traits, we
reared laboratory-born mice from both populations in a cold and warm
environment. Genetic differences in body mass between inbred lines of
New York and Brazil mice were evident at weaning (Figure 3A), with New
York mice larger than Brazil mice. These body mass differences between
populations persisted across developmental stages from 3 to 11 weeks. In
all cases, males were larger than females. Full-sibs (of the same sex)
reared at different temperatures showed no differences in body mass
(Figure 3A). At the end of the experiment, body size differences
recapitulated patterns seen across generations, with New York mice
larger than Brazil mice (Table 1; Figure 3B) and males larger than
females (Table 1; Figure 3B). The lack of plasticity in body mass was
not a result of differences in fat accumulation, as body mass index
(BMI) did not differ between populations (ANOVA, \(\chi^2\)=0.50,
\emph{P}\textgreater0.05) or environments (ANOVA, \(\chi^2\)=1.28,
\emph{P}\textgreater0.05) (Figure S4). These results suggest that
phenotypic plasticity does not play a significant role in body size
evolution of house mice.

In contrast to body mass, tail length was greatly influenced by
developmental temperature, with the first few weeks post-weaning having
the greatest influence on absolute tail length (Figure 4). Specifically,
inbred mice reared in a cold environment grew shorter tails than inbred
mice reared in a warm environment (Table 1; Figures 4, 5A). The
magnitude of this effect was striking in Brazil mice, corresponding to
5.40 mm, on average, or 7\% of the total adult tail length. Despite
developmental temperature playing a significant role in tail length,
genetic differences in tail length were evident at the end of the
experiment, with Brazil mice growing longer tails than New York mice in
both environments (Table 1; Figures 4, 5A). Thus, tail length exhibits
both genetic divergence and phenotypic plasticity. Similarly,
cold-reared mice grew shorter ears than warm-reared mice (Table 1;
Figure 5B), and the magnitude of the plastic response corresponded to
5\% of the total adult ear length in both populations. These
temperature-growth responses of the extremities were not a simple
consequence of body size differences, as body mass did not differ
between treatments (Figure 3). Overall, unlike body size, extremity
length showed significant plasticity in response to temperature, with
mice growing shorter extremities in a cold environment, consistent with
patterns of Allen's rule.

\vspace{3.5mm}

\noindent\emph{Adaptive phenotypic plasticity in extremity length}

Differences between warm- and cold-reared mice revealed a strong plastic
response to temperature in extremity length. Because plasticity is
considered adaptive when a phenotype is altered in the same direction as
natural selection (Ghalambor et al. 2007), we next asked whether
phenotypic plasticity of Brazil mice goes in the same or opposite
direction as the evolved response of New York mice. For absolute tail
length, the plastic response in Brazil house mice nearly recapitulates
the tail length of New York mice reared in a warm environment (Figure
4), highlighting an example of adaptive phenotypic plasticity.
Interestingly, the plastic response of tail length for New York mice
raised in the cold was attenuated in comparison to the plastic response
of tail length for Brazil mice (Figures 4, 5A), suggesting that New York
house mice may be closer to the phenotypic optimum or that there is a
developmental constraint on minimum tail length. Lastly, plasticity in
ear length of Brazil house mice went in the same direction as the
genetic response of New York house mice (Figure 5B), further
illustrating adaptive phenotypic plasticity. The overall degree and
direction of plasticity in extremities mirror patterns associated with
Allen's rule. In fact, the plastic response of both tail length and ear
length in Brazil mice (i.e., difference between warm Brazil trait value
and cold Brazil trait value) explains roughly 40\% of the mean
phenotypic differences we observe in wild mice (i.e., between N0 Brazil
mice and N0 New York mice).

\vspace{5mm}

\hypertarget{discussion}{%
\subsection{Discussion}\label{discussion}}

Previous studies have provided conflicting assessments of the generality
of Bergmann's and Allen's rules and have rarely combined field and
laboratory studies to identify the contribution of genetic and
non-genetic effects to phenotypic variation. Here, we focused on one
species that has recently expanded its range across many degrees of
latitude, and we studied phenotypic variation in the wild and in the
lab. Moreover, by rearing inbred mice at different temperatures we were
able to assess the contribution of phenotypic plasticity to patterns
seen in nature.

First, we found that wild house mice across North and South America
conform to Bergmann's rule and Allen's rule, as house mice are larger in
size with shortened extremities farther from the equator and at colder
temperatures. Second, heritable differences in body mass and tail and
ear length in a common environment indicated a genetic basis to
Bergmann's rule and Allen's rule, presumably reflecting thermoregulatory
adaptations. Finally, we measured the contributions of phenotypic
plasticity to these traits and found that tail and ear length are highly
plastic in response to cold temperature, while body size is not. The
plastic response in extremity length to cold temperatures appears
adaptive, matching the direction of change of extremity length seen in
wild, temperate house mice. Adaptive plasticity associated with Allen's
rule, in conjunction with strong selection for body size, likely
promoted the rapid expansion of house mice into new environments across
the Americas.

\vspace{3.5mm}

\noindent\emph{Genetic contributions to ecogeographic rules}

Parallel phenotypic clines across multiple transects provide strong
evidence for natural selection (Endler 1977). Body mass in house mice
increases as latitude increases (Figure 1A-B), resembling patterns seen
in eastern North America (Lynch 1992; Phifer-Rixey et al. 2018), South
America (Suzuki et al. 2020), and Australia (Tomlinson and Withers
2009). These clinal patterns are consistent with Bergmann's rule.
Notably, this pattern is much clearer when looking at a dataset that
includes only adult males (Figure 1B) compared to a dataset with all
animals (Figure 1A). This difference may help to explain the
discrepancies between previous meta-analyses based on museum collections
(e.g., Ashton et al. 2000; Meiri and Dayan 2003; Riemer et al. 2018).
Moreover, the substantial trait variation in Figure 1 among individuals,
even when sampled at the same latitude, provides strong motivation for
studying these traits in a common laboratory environment.

Phenotypic measurements of lab-reared mice from temperate and tropical
populations revealed a heritable basis for the difference in body mass,
with mice from a colder climate significantly larger than mice from
tropical environments (Figure 2A). This pattern was seen across
generations (Figure S2) and heritability estimates for body mass were
relatively high across both populations (Figure 2). These results agree
with previous studies that also found a heritable basis for body size
differences in house mice from eastern North America (Lynch 1992;
Phifer-Rixey et al. 2018), western North America (Ferris et al. 2021),
and South America (Suzuki et al. 2020), and suggest that there has been
strong directional selection for body size in house mice. Selection over
short time scales leading to latitudinal clines for body size has also
been shown for other non-native species, such as in the genus
\emph{Drosophila}. Specifically, body size clines in introduced species
of \emph{Drosophila} have been repeated across continents, in common
garden experiments, and through experimental evolution studies (Cavicchi
et al. 1985; Coyne and Beecham 1987; Partridge et al. 1994; James et al.
1995; Land et al. 1999; Huey et al. 2000; Gilchrist et al. 2001, 2004).
Latitudinal clines for body size have also arisen rapidly in introduced
populations of house sparrows (Johnston and Selander 1964, 1971) and
European starlings (Cardilini et al. 2016). Together, these results
suggest that introduced species may undergo strong selection while
colonizing new environments, allowing patterns conforming to Bergmann's
rule to become quickly established.

Phenotypic measurements of wild mice revealed clines for relative tail
length and relative ear length, with shorter extremities seen in mice
collected farther from the equator and at colder temperatures,
consistent with Allen's rule (Figures 1, S1). As with body size, a
clearer picture of clinal variation emerged when only considering adult
males (Figures 1D, 1F, S1D, S1F) compared to all animals. Measurements
of lab mice showed that these differences are heritable (Figures 2, S2).
Shorter tails in northern populations of house mice may minimize heat
loss and thus be an adaptation to the cold, as tail length shows a
positive correlation with temperature of the coldest month across
rodents (Alhajeri et al. 2020). Similar trends and correlations have
also been found for limb length and bill length in birds (Nudds and
Oswald 2007; Symonds and Tattersall 2010; Danner and Greenberg 2015;
Friedman et al. 2017). In addition to thermoregulatory advantages,
alternative mechanisms for Allen's rule have been postulated to explain
why longer tails are found in the tropics, such as enhanced climbing
ability with increased arboreality (Alroy 2019; Mincer and Russo 2020).
Because house mice are commensal with humans, it is unlikely that longer
tails confer a climbing advantage in the tropics.

\vspace{3.5mm}

\noindent\emph{Contributions of phenotypic plasticity to ecogeographic
rules}

Body size in house mice shows very little plasticity in response to cold
temperature (Figure 3), reaffirming that there has likely been strong
directional selection for body size in house mice. Lack of plasticity
associated with Bergmann's rule is consistent with previous studies in
laboratory mice (Sumner 1909, 1915; Ashoub 1958; Serrat et al. 2008;
Serrat 2013) and, in addition to selection, may be due to a number of
physiological factors. The environmental influence of temperature may
need to occur pre-weaning or prenatal to elicit a plastic response
(e.g., Weaver and Ingram 1969; Burness et al. 2013; Andrew et al. 2017).
Exposure to high temperatures instead of low temperatures may elicit a
plastic response in body size, as seen previously in some endotherms
(Ashoub 1958; Gordon 2012; Burness et al. 2013; Andrew et al. 2017).

Unlike Bergmann's rule, Allen's rule can be generated via developmental
phenotypic plasticity, as extremity length is highly sensitive to
ambient temperature in both mammals and birds (Serrat 2014; Tattersall
et al. 2017). Our results in house mice agree with previous studies in
mammals, with mice growing shorter tails and ears in a cold environment
(Figure 5) (Ogle and Mills 1933; Harland 1960; Chevillard et al. 1963;
Weaver and Ingram 1969). In laboratory mice, temperature directly
affects the growth of cartilage in both tails and ears, influencing
extremity length (Serrat et al. 2008). Furthermore, the widespread
patterns of tail length plasticity in response to cold are also
recapitulated at the skeletal level, with both the length and number of
caudal vertebrae decreasing in response to cold temperatures in mice
(Barnett 1965; Noel and Wright 1970; Thorington Jr 1970; Al-Hilli and
Wright 1983). Although we did not measure skeletal differences between
New York and Brazil mice, it seems likely that the tail length
plasticity we observed is a result of plasticity in both number and
length of individual caudal vertebrae. Moreover, ear length shows the
greatest plasticity in both populations, with both New York and Brazil
mice growing shorter ears in the cold. The pronounced plastic response
of ears compared to tails may indicate that smaller appendages
consisting entirely of cartilage are less developmentally canalized.
Less constraint associated with extremities may also underly the highly
plastic nature of Allen's rule compared to Bergmann's rule. This is
illustrated by tail length and ear length plasticity accounting for
roughly 40\% of the observed differences among wild New York and Brazil
house mice.

\vspace{3.5mm}

\noindent\emph{Adaptive phenotypic plasticity and Allen's rule}

Phenotypic plasticity is adaptive when it aligns with the direction of
selection, moving traits closer to the local phenotypic optima (Baldwin
1896; West-Eberhard 2003; Ghalambor et al. 2007). We found evidence for
adaptive phenotypic plasticity underlying Allen's rule, as plasticity
produced shorter ears and tails in cold environments. We also observed
an attenuated plastic response for tail length in New York house mice
compared to Brazil house mice, suggesting that New York mice are closer
to the phenotypic optimum and are better adapted to colder environments.
Overall, plasticity in house mouse extremities mirrors general
evolutionary patterns of shorter extremity lengths in colder climates
and may play an important role in generating Allen's rule.

There are two ways by which adaptive phenotypic plasticity can
facilitate the colonization of new environments. Adaptive plasticity can
incompletely move the trait value closer to the phenotypic optimum, with
directional selection refining the trait value, leading to subsequent
genetic changes (Price et al. 2003; Ghalambor et al. 2007).
Alternatively, adaptive plasticity can slow or impede evolution by
moving individuals completely to the phenotypic optimum, shielding
genetic variation from natural selection (Price et al. 2003; Ghalambor
et al. 2007). We find evidence for the first scenario for both tail and
ear length in house mice. Specifically, both genetic and plastic
contributions generate shorter tails and ears in colder environments.
Despite the plastic response of extremity length in Brazil mice
explaining roughly 40\% of the mean phenotypic differences observed in
wild mice, we see clear evidence of genetic differences in tail length
and ear length between New York and Brazil house mice. This suggests
that phenotypic plasticity moves extremity length close to the local
optimum but does not shield it from subsequent selection. Overall,
adaptive phenotypic plasticity in addition to strong, directional
selection underlying Bergmann's rule, likely facilitated the rapid
expansion of house mice into new environments across the Americas.

\vspace{5mm}

\hypertarget{acknowledgements}{%
\subsection{Acknowledgements}\label{acknowledgements}}

\noindent We thank Michael Sheehan and Felipe Martins for collecting
wild mice, and we thank Kathleen Ferris, Gabriela Heyer, Dana Lin,
Felipe Martins, Megan Phifer-Rixey, Michael Sheehan, and Taichi Suzuki
for help with mouse husbandry. We also thank Jesse Alston, Libby
Beckman, Sylvia Durkin, Emilie Richards, Michelle St.~John, Molly
Womack, Daniel Bolnick, David Lowry, and two anonymous reviewers for
their constructive feedback that improved the manuscript. M.A.B was
supported by a National Science Foundation Graduate Research Fellowship
(DGE 1106400), Junea W. Kelly Museum of Vertebrate Zoology Graduate
Fellowship, and a UC Berkeley Philomathia Graduate Fellowship. This work
was supported by graduate student research funds from the Museum of
Vertebrate Zoology and Department of Integrative Biology to M.A.B and an
NIH grant to M.W.N. (R01GM127468).

\vspace{5mm}

\hypertarget{references}{%
\subsection{References}\label{references}}

\setlength{\parindent}{-0.25in}
\setlength{\leftskip}{0.25in}

\noindent

\hypertarget{refs}{}
\begin{CSLReferences}{0}{0}
\leavevmode\hypertarget{ref-Alhajeri2020}{}%
Alhajeri, B. H., Y. Fourcade, N. S. Upham, and H. Alhaddad. 2020. A
global test of {Allen's} rule in rodents. Global Ecology and
Biogeography 29:2248--2260.

\leavevmode\hypertarget{ref-Alhajeri2016}{}%
Alhajeri, B. H., and S. J. Steppan. 2016. Association between climate
and body size in rodents: A phylogenetic test of {Bergmann's} rule.
Mammalian Biology 81:219--225.

\leavevmode\hypertarget{ref-Al-Hilli1983}{}%
Al-Hilli, F., and E. Wright. 1983. The effects of changes in the
environmental temperature on the growth of bone in the mouse.
Radiological and morphological study. British Journal of Experimental
Pathology 64:43.

\leavevmode\hypertarget{ref-Alho2011}{}%
Alho, J., G. Herczeg, A. Laugen, K. Räsänen, A. Laurila, and J. Merilä.
2011. Allen's rule revisited: Quantitative genetics of extremity length
in the common frog along a latitudinal gradient. Journal of Evolutionary
Biology 24:59--70.

\leavevmode\hypertarget{ref-Allaire2021}{}%
Allaire, J., Y. Xie, J. McPherson, J. Luraschi, K. Ushey, A. Atkins, H.
Wickham, et al. 2021. Rmarkdown: Dynamic documents for {R}.

\leavevmode\hypertarget{ref-Allen1877}{}%
Allen, J. A. 1877. The influence of physical conditions in the genesis
of species. Radical Review 1:108--140.

\leavevmode\hypertarget{ref-Alroy2019}{}%
Alroy, J. 2019. Small mammals have big tails in the tropics. Global
Ecology and Biogeography 28:1042--1050.

\leavevmode\hypertarget{ref-Andrew2017}{}%
Andrew, S., L. Hurley, M. Mariette, and S. Griffith. 2017. Higher
temperatures during development reduce body size in the zebra finch in
the laboratory and in the wild. Journal of Evolutionary Biology
30:2156--2164.

\leavevmode\hypertarget{ref-Ashoub1958}{}%
Ashoub, M. E.-R. 1958. Effect of two extreme temperatures on growth and
tail-length of mice. Nature 181:284--284.

\leavevmode\hypertarget{ref-Ashton2002}{}%
Ashton, K. G. 2002. Patterns of within-species body size variation of
birds: Strong evidence for {Bergmann's} rule. Global Ecology and
Biogeography 11:505--523.

\leavevmode\hypertarget{ref-Ashton2000}{}%
Ashton, K. G., M. C. Tracy, and A. de Queiroz. 2000. Is {Bergmann's}
rule valid for mammals? The American Naturalist 156:390--415.

\leavevmode\hypertarget{ref-Baldwin1896}{}%
Baldwin, J. M. 1896. A new factor in evolution. The American Naturalist
30:441--451.

\leavevmode\hypertarget{ref-Barnett1965}{}%
Barnett, S. 1965. Genotype and environment in tail length in mice.
Quarterly Journal of Experimental Physiology and Cognate Medical
Sciences: Translation and Integration 50:417--429.

\leavevmode\hypertarget{ref-Barnett1984}{}%
Barnett, S., and R. Dickson. 1984. Changes among wild house mice ({Mus}
musculus) bred for ten generations in a cold environment, and their
evolutionary implications. Journal of Zoology 203:163--180.

\leavevmode\hypertarget{ref-Bates2015}{}%
Bates, D., M. Mächler, B. Bolker, and S. Walker. 2015. Fitting linear
mixed-effects models using {lme4}. Journal of Statistical Software
67:1--48.

\leavevmode\hypertarget{ref-Ben-Shachar2020}{}%
Ben-Shachar, M. S., D. Lüdecke, and D. Makowski. 2020. {e}ffectsize:
Estimation of effect size indices and standardized parameters. Journal
of Open Source Software 5:2815.

\leavevmode\hypertarget{ref-Bergmann1847}{}%
Bergmann, C. 1847. {Ü}ber die verh{ä}ltnisse der w{ä}rme{ö}konomie der
thiere zu ihrer gr{ö}sse. Gottinger Studien 3:595--708.

\leavevmode\hypertarget{ref-Betti2015}{}%
Betti, L., S. J. Lycett, N. von Cramon-Taubadel, and O. M. Pearson.
2015. Are human hands and feet affected by climate? A test of {Allen's}
rule. American Journal of Physical Anthropology 158:132--140.

\leavevmode\hypertarget{ref-Blackburn2004}{}%
Blackburn, T. M., and B. A. Hawkins. 2004. Bergmann's rule and the
mammal fauna of northern {North} {America}. Ecography 27:715--724.

\leavevmode\hypertarget{ref-Brown1969}{}%
Brown, J. H., and A. K. Lee. 1969. Bergmann's rule and climatic
adaptation in woodrats ({Neotoma}). Evolution 329--338.

\leavevmode\hypertarget{ref-Burness2013}{}%
Burness, G., J. R. Huard, E. Malcolm, and G. J. Tattersall. 2013.
Post-hatch heat warms adult beaks: Irreversible physiological plasticity
in {Japanese} quail. Proceedings of the Royal Society B: Biological
Sciences 280:20131436.

\leavevmode\hypertarget{ref-Cardilini2016}{}%
Cardilini, A. P., K. L. Buchanan, C. D. Sherman, P. Cassey, and M. R.
Symonds. 2016. Tests of ecogeographical relationships in a non-native
species: What rules avian morphology? Oecologia 181:783--793.

\leavevmode\hypertarget{ref-Cavicchi1985}{}%
Cavicchi, S., D. Guerra, G. Giorgi, and C. Pezzoli. 1985.
Temperature-related divergence in experimental populations of
{Drosophila} melanogaster. I. Genetic and developmental basis of wing
size and shape variation. Genetics 109:665--689.

\leavevmode\hypertarget{ref-Chevillard1963}{}%
Chevillard, L., R. Portet, and C. M. 1963. Growth rate of rats born and
reared at 5 and 30 c. Federation Proceedings 22:699--703.

\leavevmode\hypertarget{ref-Conover1995}{}%
Conover, D. O., and E. T. Schultz. 1995. Phenotypic similarity and the
evolutionary significance of countergradient variation. Trends in
Ecology \& Evolution 10:248--252.

\leavevmode\hypertarget{ref-Constable2010}{}%
Constable, H., R. Guralnick, J. Wieczorek, C. Spencer, A. T. Peterson,
V. S. Committee, and others. 2010. VertNet: A new model for biodiversity
data sharing. PLoS Biology 8:e1000309.

\leavevmode\hypertarget{ref-Coyne1987}{}%
Coyne, J. A., and E. Beecham. 1987. Heritability of two morphological
characters within and among natural populations of {Drosophila}
melanogaster. Genetics 117:727--737.

\leavevmode\hypertarget{ref-Danner2015}{}%
Danner, R. M., and R. Greenberg. 2015. A critical season approach to
{Allen's} rule: Bill size declines with winter temperature in a cold
temperate environment. Journal of Biogeography 42:114--120.

\leavevmode\hypertarget{ref-DesMarais2013}{}%
Des Marais, D. L., K. M. Hernandez, and T. E. Juenger. 2013.
Genotype-by-environment interaction and plasticity: Exploring genomic
responses of plants to the abiotic environment. Annual Review of
Ecology, Evolution, and Systematics 44:5--29.

\leavevmode\hypertarget{ref-Endler1977}{}%
Endler, J. A. 1977. Geographic variation, speciation, and clines.
Princeton University Press.

\leavevmode\hypertarget{ref-Falconer1996}{}%
Falconer, D. S., and T. F. C. Mackay. 1996. Introduction to quantitative
genetics. Pearson.

\leavevmode\hypertarget{ref-Ferris2021}{}%
Ferris, K. G., A. S. Chavez, T. A. Suzuki, E. J. Beckman, M.
Phifer-Rixey, K. Bi, and M. W. Nachman. 2021. The genomics of rapid
climatic adaptation and parallel evolution in {North} {American} house
mice. PLoS Genetics 17:e1009495.

\leavevmode\hypertarget{ref-Fick2017}{}%
Fick, S. E., and R. J. Hijmans. 2017. WorldClim 2: New 1-km spatial
resolution climate surfaces for global land areas. International journal
of climatology 37:4302--4315.

\leavevmode\hypertarget{ref-Fooden1999}{}%
Fooden, J., and G. H. Albrecht. 1999. Tail-length evolution in
fascicularis-group macaques ({Cercopithecidae}: macaca). International
Journal of Primatology 20:431--440.

\leavevmode\hypertarget{ref-Foster2013}{}%
Foster, F., and M. Collard. 2013. A reassessment of {Bergmann's} rule in
modern humans. PloS One 8:e72269.

\leavevmode\hypertarget{ref-Fox2019}{}%
Fox, J., and S. Weisberg. 2019. An {R} companion to applied regression
(Third.). Sage, Thousand Oaks {CA}.

\leavevmode\hypertarget{ref-Freckleton2003}{}%
Freckleton, R. P., P. H. Harvey, and M. Pagel. 2003. Bergmann's rule and
body size in mammals. The American Naturalist 161:821--825.

\leavevmode\hypertarget{ref-Friedman2017}{}%
Friedman, N. R., L. Harmáčková, E. P. Economo, and V. Remeš. 2017.
Smaller beaks for colder winters: Thermoregulation drives beak size
evolution in {Australasian} songbirds. Evolution 71:2120--2129.

\leavevmode\hypertarget{ref-Geist1987}{}%
Geist, V. 1987. Bergmann's rule is invalid. Canadian Journal of Zoology
65:1035--1038.

\leavevmode\hypertarget{ref-Ghalambor2007}{}%
Ghalambor, C. K., J. K. McKay, S. P. Carroll, and D. N. Reznick. 2007.
Adaptive versus non-adaptive phenotypic plasticity and the potential for
contemporary adaptation in new environments. Functional Ecology
21:394--407.

\leavevmode\hypertarget{ref-Gilchrist2004a}{}%
Gilchrist, G. W., R. B. Huey, J. Balanyà, M. Pascual, and L. Serra.
2004. A time series of evolution in action: A latitudinal cline in wing
size in {South} {American} {Drosophila} subobscura. Evolution
58:768--780.

\leavevmode\hypertarget{ref-Gilchrist2001}{}%
Gilchrist, G. W., R. B. Huey, and L. Serra. 2001. Rapid evolution of
wing size clines in {Drosophila} subobscura. Genetica 273--286.

\leavevmode\hypertarget{ref-Gillespie1989}{}%
Gillespie, J. H., and M. Turelli. 1989. Genotype-environment
interactions and the maintenance of polygenic variation. Genetics
121:129--138.

\leavevmode\hypertarget{ref-Gohli2016}{}%
Gohli, J., and K. L. Voje. 2016. An interspecific assessment of
{Bergmann's} rule in 22 mammalian families. BMC Evolutionary Biology
16:1--12.

\leavevmode\hypertarget{ref-Gomulkiewicz1992}{}%
Gomulkiewicz, R., and M. Kirkpatrick. 1992. Quantitative genetics and
the evolution of reaction norms. Evolution 46:390--411.

\leavevmode\hypertarget{ref-Gordon2012}{}%
Gordon, C. 2012. Thermal physiology of laboratory mice: Defining
thermoneutrality. Journal of Thermal Biology 37:654--685.

\leavevmode\hypertarget{ref-Griffing1974}{}%
Griffing, J. P. 1974. Body measurements of black-tailed jackrabbits of
southeastern {New} {Mexico} with implications of {Allen's} rule. Journal
of Mammalogy 55:674--678.

\leavevmode\hypertarget{ref-Harland1960}{}%
Harland, S. 1960. Effect of temperature on growth in weight and
tail-length of inbred and hybrid mice. Nature 186:446.

\leavevmode\hypertarget{ref-Harpak2021}{}%
Harpak, A., and M. Przeworski. 2021. The evolution of group differences
in changing environments. PLoS Biology 19:e3001072.

\leavevmode\hypertarget{ref-Huey2000}{}%
Huey, R. B., G. W. Gilchrist, M. L. Carlson, D. Berrigan, and L. Serra.
2000. Rapid evolution of a geographic cline in size in an introduced
fly. Science 287:308--309.

\leavevmode\hypertarget{ref-Husby2011}{}%
Husby, A., S. M. Hille, and M. E. Visser. 2011. Testing mechanisms of
{Bergmann's} rule: Phenotypic decline but no genetic change in body size
in three passerine bird populations. The American Naturalist
178:202--213.

\leavevmode\hypertarget{ref-Huxley1939}{}%
Huxley, J. S. 1939. Clines: An auxiliary method in taxonomy. Bijdragen
tot de Dierkunde 27:491--520.

\leavevmode\hypertarget{ref-James1995}{}%
James, A. C., R. Azevedo, and L. Partridge. 1995. Cellular basis and
developmental timing in a size cline of {Drosophila} melanogaster.
Genetics 140:659--666.

\leavevmode\hypertarget{ref-James1970}{}%
James, F. C. 1970. Geographic size variation in birds and its
relationship to climate. Ecology 51:365--390.

\leavevmode\hypertarget{ref-James1983}{}%
---------. 1983. Environmental component of morphological
differentiation in birds. Science 221:184--186.

\leavevmode\hypertarget{ref-Johnston1964}{}%
Johnston, R. F., and R. K. Selander. 1964. House sparrows: Rapid
evolution of races in {North} {America}. Science 144:548--550.

\leavevmode\hypertarget{ref-Johnston1971}{}%
---------. 1971. Evolution in the house sparrow. II. Adaptive
differentiation in {North} {American} populations. Evolution 1--28.

\leavevmode\hypertarget{ref-Laiolo2001}{}%
Laiolo, P., and A. Rolando. 2001. Ecogeographic correlates of
morphometric variation in the red-billed chough {Pyrrhocorax}
pyrrhocorax and the alpine chough {Pyrrhocorax} graculus. Ibis
143:602--616.

\leavevmode\hypertarget{ref-Land1999}{}%
Land, J. V. `t., P. V. Putten, and W. V. Delden. 1999. Latitudinal
variation in wild populations of {Drosophila} melanogaster:
Heritabilities and reaction norms. Journal of Evolutionary Biology
12:222--232.

\leavevmode\hypertarget{ref-Luxfcdecke2021}{}%
Lüdecke, D., M. S. Ben-Shachar, I. Patil, P. Waggoner, and D. Makowski.
2021. Assessment, testing and comparison of statistical models using
{R}. Journal of Open Source Software 6:3112.

\leavevmode\hypertarget{ref-Lynch1992}{}%
Lynch, C. B. 1992. Clinal variation in cold adaptation in {Mus}
domesticus: Verification of predictions from laboratory populations. The
American Naturalist 139:1219--1236.

\leavevmode\hypertarget{ref-Lynch1998}{}%
Lynch, M., B. Walsh, and others. 1998. Genetics and analysis of
quantitative traits. Sinauer Sunderland, MA.

\leavevmode\hypertarget{ref-Mayr1956}{}%
Mayr, E. 1956. Geographical character gradients and climatic adaptation.
Evolution 10:105--108.

\leavevmode\hypertarget{ref-McNab1971}{}%
McNab, B. K. 1971. On the ecological significance of {Bergmann's} rule.
Ecology 52:845--854.

\leavevmode\hypertarget{ref-Meiri2003}{}%
Meiri, S., and T. Dayan. 2003. On the validity of {Bergmann's} rule.
Journal of Biogeography 30:331--351.

\leavevmode\hypertarget{ref-Millien2006}{}%
Millien, V., S. Kathleen Lyons, L. Olson, F. A. Smith, A. B. Wilson, and
Y. Yom-Tov. 2006. Ecotypic variation in the context of global climate
change: Revisiting the rules. Ecology Letters 9:853--869.

\leavevmode\hypertarget{ref-Mincer2020}{}%
Mincer, S. T., and G. A. Russo. 2020. Substrate use drives the
macroevolution of mammalian tail length diversity. Proceedings of the
Royal Society B 287:20192885.

\leavevmode\hypertarget{ref-Noel1970}{}%
Noel, J. F., and E. Wright. 1970. The effect of environmental
temperature on the growth of vertebrae in the tail of the mouse.
Development 24:405--410.

\leavevmode\hypertarget{ref-Nudds2007}{}%
Nudds, R., and S. Oswald. 2007. An interspecific test of {Allen's} rule:
Evolutionary implications for endothermic species. Evolution:
International Journal of Organic Evolution 61:2839--2848.

\leavevmode\hypertarget{ref-Ogle1933}{}%
Ogle, C., and C. Mills. 1933. Animal adaptation to environmental
temperature conditions. American Journal of Physiology-Legacy Content
103:606--612.

\leavevmode\hypertarget{ref-Olejnik2003}{}%
Olejnik, S., and J. Algina. 2003. Generalized eta and omega squared
statistics: Measures of effect size for some common research designs.
Psychological methods 8:434.

\leavevmode\hypertarget{ref-Olson2009}{}%
Olson, V. A., R. G. Davies, C. D. L. Orme, G. H. Thomas, S. Meiri, T. M.
Blackburn, K. J. Gaston, et al. 2009. Global biogeography and ecology of
body size in birds. Ecology Letters 12:249--259.

\leavevmode\hypertarget{ref-Ozgul2010}{}%
Ozgul, A., D. Z. Childs, M. K. Oli, K. B. Armitage, D. T. Blumstein, L.
E. Olson, S. Tuljapurkar, et al. 2010. Coupled dynamics of body mass and
population growth in response to environmental change. Nature
466:482--485.

\leavevmode\hypertarget{ref-Ozgul2009}{}%
Ozgul, A., S. Tuljapurkar, T. G. Benton, J. M. Pemberton, T. H.
Clutton-Brock, and T. Coulson. 2009. The dynamics of phenotypic change
and the shrinking sheep of {St.} kilda. Science 325:464--467.

\leavevmode\hypertarget{ref-Partridge1994}{}%
Partridge, L., B. Barrie, K. Fowler, and V. French. 1994. Evolution and
development of body size and cell size in {Drosophila} melanogaster in
response to temperature. Evolution 48:1269--1276.

\leavevmode\hypertarget{ref-Phifer-Rixey2018}{}%
Phifer-Rixey, M., K. Bi, K. G. Ferris, M. J. Sheehan, D. Lin, K. L.
Mack, S. M. Keeble, et al. 2018. The genomic basis of environmental
adaptation in house mice. PLoS Genetics 14:e1007672.

\leavevmode\hypertarget{ref-Phifer-Rixey2015}{}%
Phifer-Rixey, M., and M. W. Nachman. 2015. The natural history of model
organisms: Insights into mammalian biology from the wild house mouse
{Mus} musculus. Elife 4:e05959.

\leavevmode\hypertarget{ref-Price2003}{}%
Price, T. D., A. Qvarnström, and D. E. Irwin. 2003. The role of
phenotypic plasticity in driving genetic evolution. Proceedings of the
Royal Society of London. Series B: Biological Sciences 270:1433--1440.

\leavevmode\hypertarget{ref-Riemer2018}{}%
Riemer, K., R. P. Guralnick, and E. P. White. 2018. No general
relationship between mass and temperature in endothermic species. Elife
7:e27166.

\leavevmode\hypertarget{ref-Romano2020}{}%
Romano, A., R. Séchaud, and A. Roulin. 2020. Geographical variation in
bill size provides evidence for {Allen's} rule in a cosmopolitan raptor.
Global Ecology and Biogeography 29:65--75.

\leavevmode\hypertarget{ref-Ruff2002}{}%
Ruff, C. 2002. Variation in human body size and shape. Annual Review of
Anthropology 31:211--232.

\leavevmode\hypertarget{ref-Ruff1994}{}%
Ruff, C. B. 1994. Morphological adaptation to climate in modern and
fossil hominids. American Journal of Physical Anthropology 37:65--107.

\leavevmode\hypertarget{ref-Rutledge1973}{}%
Rutledge, J. J., E. Eisen, and J. Legates. 1973. An experimental
evaluation of genetic correlation. Genetics 75:709--726.

\leavevmode\hypertarget{ref-Scholander1955}{}%
Scholander, P. F. 1955. Evolution of climatic adaptation in homeotherms.
Evolution 15--26.

\leavevmode\hypertarget{ref-Serrat2013}{}%
Serrat, M. A. 2013. Allen's rule revisited: Temperature influences bone
elongation during a critical period of postnatal development. The
Anatomical Record 296:1534--1545.

\leavevmode\hypertarget{ref-Serrat2014}{}%
---------. 2014. Environmental temperature impact on bone and cartilage
growth. Comprehensive Physiology 4:621--655.

\leavevmode\hypertarget{ref-Serrat2008}{}%
Serrat, M. A., D. King, and C. O. Lovejoy. 2008. Temperature regulates
limb length in homeotherms by directly modulating cartilage growth.
Proceedings of the National Academy of Sciences 105:19348--19353.

\leavevmode\hypertarget{ref-Sumner1909}{}%
Sumner, F. B. 1909. Some effects of external conditions upon the white
mouse. The Journal of Experimental Zoology 7:97--155.

\leavevmode\hypertarget{ref-Sumner1915}{}%
---------. 1915. Some studies of environmental influence, heredity,
correlation and growth, in the white mouse. The Journal of Experimental
Zoology 18:325.

\leavevmode\hypertarget{ref-Suzuki2020}{}%
Suzuki, T. A., F. M. Martins, Phifer-Rixey Megan, and M. W. Nachman.
2020. The gut microbiota and {Bergmann's} rule in wild house mice.
Molecular Ecology 29:2300--2311.

\leavevmode\hypertarget{ref-Symonds2010}{}%
Symonds, M. R., and G. J. Tattersall. 2010. Geographical variation in
bill size across bird species provides evidence for {Allen's} rule. The
American Naturalist 176:188--197.

\leavevmode\hypertarget{ref-Tattersall2017}{}%
Tattersall, G. J., B. Arnaout, and M. R. Symonds. 2017. The evolution of
the avian bill as a thermoregulatory organ. Biological Reviews
92:1630--1656.

\leavevmode\hypertarget{ref-Teplitsky2008}{}%
Teplitsky, C., J. A. Mills, J. S. Alho, J. W. Yarrall, and J. Merilä.
2008. Bergmann's rule and climate change revisited: Disentangling
environmental and genetic responses in a wild bird population.
Proceedings of the National Academy of Sciences 105:13492--13496.

\leavevmode\hypertarget{ref-Thorington1970}{}%
Thorington Jr, R. W. 1970. Lability of tail length of the white-footed
mouse, {Peromyscus} leucopus noveboracensis. Journal of Mammalogy
51:52--59.

\leavevmode\hypertarget{ref-Tomlinson2009}{}%
Tomlinson, S., and P. C. Withers. 2009. Biogeographical effects on body
mass of native {Australian} and introduced mice, {Pseudomys}
hermannsburgensis and {Mus} domesticus: An inquiry into {Bergmann's}
rule. Australian Journal of Zoology 56:423--430.

\leavevmode\hypertarget{ref-Via1985}{}%
Via, S., and R. Lande. 1985. Genotype-environment interaction and the
evolution of phenotypic plasticity. Evolution 39:505--522.

\leavevmode\hypertarget{ref-Weaver1969}{}%
Weaver, M. E., and D. L. Ingram. 1969. Morphological changes in swine
associated with environmental temperature. Ecology 50:710--713.

\leavevmode\hypertarget{ref-West-Eberhard2003}{}%
West-Eberhard, M. J. 2003. Developmental plasticity and evolution.
Oxford University Press.

\leavevmode\hypertarget{ref-Wickham2019}{}%
Wickham, H., M. Averick, J. Bryan, W. Chang, L. D. McGowan, R. François,
G. Grolemund, et al. 2019. Welcome to the {tidyverse}. Journal of Open
Source Software 4:1686.

\leavevmode\hypertarget{ref-Yang2010}{}%
Yang, J., B. Benyamin, B. P. McEvoy, S. Gordon, A. K. Henders, D. R.
Nyholt, P. A. Madden, et al. 2010. Common SNPs explain a large
proportion of the heritability for human height. Nature genetics
42:565--569.

\leavevmode\hypertarget{ref-Yom-Tov1986}{}%
Yom-Tov, Y., and H. Nix. 1986. Climatological correlates for body size
of five species of {Australian} mammals. Biological Journal of the
Linnean Society 29:245--262.

\end{CSLReferences}

\setlength{\parindent}{0in}
\setlength{\leftskip}{0in}

\hypertarget{tables}{%
\subsection{Tables}\label{tables}}

\textbf{Table 1. Results of linear mixed models investigating the
effects of sex, population, environment, and their interaction on body
mass and extremity length in house mice.}

\newpage

\hypertarget{figures}{%
\subsection{Figures}\label{figures}}

\includegraphics{../results/figures/VertNet_relative.pdf}

\textbf{Figure 1. Bergmann's rule and Allen's rule in house mice from
North and South America.} Associations between body mass (A-B), tail
length (C-D), ear length (E-F), and absolute latitude across wild-caught
North and South American house mice. Tail length and ear length are
plotted relative to body mass for each individual. Individuals are
represented as individual points, with males depicted in black and
females depicted in white. Results from Spearman correlations are
presented in each plot, along with sample sizes. For clarity, standard
error shading is omitted from linear regression lines associated with
the VertNet Metadata (panels A, C, and E).

\newpage

\includegraphics{../results/figures/N2N3_h2.pdf}

\textbf{Figure 2. Heritability (\emph{h\textsuperscript{2}}) estimates
for body mass and extremity length in New York and Brazil house mice.}
Midparent-offspring regressions were performed on body mass (A),
relative tail length (B), and relative ear length (C). Midparent values
were calculated as the mean trait value between mothers and fathers for
13 families in each Brazil and New York populations. Heritabilities
(\emph{h\textsuperscript{2}} ± standard error) were estimated as slopes
of midparent-offspring regressions. Significance of each regression was
assessed with ANOVAs and is represented with an asterisk. Sample sizes:
New York: \emph{n} = 22; Brazil: \emph{n} = 26.

\newpage

\includegraphics{../results/figures/Weekly_RXN_BW.pdf}

\textbf{Figure 3. Genetic differences and very little plasticity in body
mass among New York and Brazil house mice.} A) Body mass growth
trajectories across environments and 11 weeks of development in New York
(blue) and Brazil (gold) house mice. Individuals are plotted as
individual points, with cold-reared mice denoted as open circles and
warm-reared mice denoted as filled circles (New York: \emph{n} =
20/treatment; Brazil: \emph{n} = 20/treatment). Population means are
depicted as smoothed regression fits, with cold mice denoted as dashed
lines and warm mice denoted as solid lines. Each line is accompanied
with standard error shading. B) Individuals at 11 weeks of age are
represented as individual points (New York: \emph{n} = 20/treatment;
Brazil: \emph{n} = 20/treatment), and boxplots indicate the 25th,
median, and 75th quartiles. Dashed lines connect median values of each
genotype across warm and cold environments. Results from linear mixed
models are presented in the upper right corner (*P\textless0.05; Table
1).

\newpage

\includegraphics{../results/figures/Weekly_Tails.pdf}

\textbf{Figure 4. Tail length is highly influenced by cold temperature
across development.} Absolute tail length growth trajectories across
environments in New York (blue) and Brazil (gold) house mice.
Individuals are plotted as individual points, with cold-reared mice
denoted as open circles and warm-reared mice denoted as filled circles
(New York: \emph{n} = 20/treatment; Brazil: \emph{n} = 20/treatment).
Population means are depicted as smoothed regression fits, with cold
mice denoted as dashed lines and warm mice are denoted as solid lines.
Each line is accompanied with standard error shading. The same
individuals depicted here are also depicted in Figure 5.

\newpage

\includegraphics{../results/figures/RXNs_Extremities_relative.pdf}

\textbf{Figure 5. Adaptive phenotypic plasticity in extremity length
among New York and Brazil house mice.} Tail length (A) and ear length
(B) are plotted relative to body mass for each individual. Individuals
are represented as individual points (A) New York: \emph{n} =
20/treatment; Brazil: \emph{n} = 20/treatment; B) New York: \emph{n} =
20/treatment; Brazil: \emph{n} = 19/treatment). Boxplots indicate the
25th, median, and 75th quartiles. Dashed lines connect median values of
each genotype across warm and cold environments. Both sexes were
combined for simplicity. Results from linear mixed models are presented
in the upper right corner (*\emph{P}\textless0.05; Table 1). The same
individuals depicted here in (A) are also depicted in Figure 4.

\newpage

\hypertarget{supplemental-tables}{%
\subsection{Supplemental Tables}\label{supplemental-tables}}

\textbf{Table S1. Results of linear models investigating the
relationship between body mass, extremity length, and absolute
latitude.} Sex was included as a covariate in each model.

\newpage

\textbf{Table S2. Results of linear mixed models investigating the
effects of sex, population, generation, and their interaction on body
mass and extremity length in house mice.} Results of analysis are
plotted in Figure S2.

\newpage

\hypertarget{supplemental-figures}{%
\subsection{Supplemental Figures}\label{supplemental-figures}}

\includegraphics{../results/figures/VertNet_meantemp.pdf}

\textbf{Figure S1. Bergmann's rule and Allen's rule in house mice from
North and South America.} Associations between body mass (A-B), tail
length (C-D), ear length (E-F), and mean annual temperature
(\textsuperscript{o}C) across wild-caught North and South American house
mice. Tail length and ear length are plotted relative to body mass for
each individual. Individuals are represented as individual points, with
males depicted in black and females depicted in white. Results from
Spearman correlations are presented in each plot, along with sample
sizes. For clarity, standard error shading is omitted from linear
regression lines associated with the VertNet Metadata (panels A, C, and
E).

\newpage

\includegraphics{../results/figures/Generations_relative_N0-N4.pdf}

\textbf{Figure S2. Body mass and extremity length differences among
populations persist over generations in a common lab environment.}
Differences in body mass (A), tail length (B), and ear length (C)
between New York mice (blue) and Brazil mice (gold) across generations.
Tail length and ear length are plotted relative to body mass for each
individual. Population-level data are depicted as boxplots overlayed on
density plots, with boxplot vertical lines denoting 1.5x the
inerquartile range. Individuals are represented as individual points.
Results from linear models are presented in the upper right corner of
each panel (*\emph{P}\textless0.05; Table S2). Abbreviations: N0 =
generation 0 (wild mice); N1 = generation 1; N2-N4 = generations 2-4.
Sample sizes: (A) Females (New York: \emph{n} = 146; Brazil: \emph{n} =
157); Males (New York: \emph{n} = 140; Brazil: \emph{n} = 133); (B)
Females (New York: \emph{n} = 145; Brazil: \emph{n} = 156); Males (New
York: \emph{n} = 140; Brazil: \emph{n} = 125); (C) Females (New York:
\emph{n} = 145; Brazil: \emph{n} = 156); Males (New York: \emph{n} =
139; Brazil: \emph{n} = 128).

\newpage

\includegraphics{../results/figures/FigS3_h2.pdf}

\textbf{Figure S3. Maternal- and paternal-offspring heritability
(\emph{h\textsuperscript{2}}) estimates for body mass and extremity
length in New York and Brazil house mice.} Maternal-offspring and
paternal-offspring regressions were performed on body mass (A), relative
tail length (B), and relative ear length (C). Heritabilities
(\emph{h\textsuperscript{2}} ± standard error) were estimated as twice
the slope and standard error of single-parent offspring regressions.
Significance of each regression were assessed with ANOVAs and are
represented with an asterisk. Sample sizes: New York: \emph{n} = 25;
Brazil: \emph{n} = 30.

\newpage

\includegraphics{../results/figures/RXNs_BMI.pdf}

\textbf{Figure S4. No genetic differences or plasticity in body mass
index (BMI) among New York mice and Brazil mice.} Individuals are
represented as individual points (New York: n** = 20/treatment; Brazil:
\emph{n} = 20/treatment), and boxplots indicate the 25th, median, and
75th quartiles. Dashed lines connect median values of each genotype
across warm and cold environments. The same individuals depicted here
are also depicted in Figure 3.

\end{document}
